\documentclass[a4paper,oneside,onecolumn]{article}
\usepackage[utf8x]{inputenc}
\usepackage[T1]{fontenc}
\usepackage{mathtools}
\usepackage{color}
\usepackage{siunitx}
\usepackage{microtype}
\usepackage{fancyhdr}
\usepackage{amsmath}
\usepackage{listings}
\usepackage[T1]{fontenc}
\usepackage{amssymb}
\usepackage{}
\usepackage[head=12pt,hmargin=2.5cm,vmargin=2.5cm]{geometry}

\renewcommand*\lstlistingname{Code Snippet}

\definecolor{codegreen}{rgb}{0,0.6,0}
\definecolor{codegray}{rgb}{0.5,0.5,0.5}
\definecolor{codepurple}{rgb}{0.58,0,0.82}
\definecolor{backcolour}{rgb}{0.95,0.95,0.92}

\lstdefinestyle{mystyle}{
    backgroundcolor=\color{backcolour},   
    commentstyle=\color{codegreen},
    keywordstyle=\color{magenta},
    numberstyle=\tiny\color{codegray},
    stringstyle=\color{codepurple},
    basicstyle=\footnotesize,
    breakatwhitespace=false,         
    breaklines=true,                 
    captionpos=b,                    
    keepspaces=true,                 
    numbers=left,                    
    numbersep=5pt,                  
    showspaces=false,                
    showstringspaces=false,
    showtabs=false,                  
    tabsize=2
}
 
\lstset{style=mystyle}

\DisableLigatures[>,<]{encoding = T1,family=tt*} %

\definecolor{codegray}{gray}{0.9}
\newcommand{\code}[1]{\colorbox{codegray}{\texttt{#1}}}

\pagestyle{fancy}

\begin{document}

\title{Lab 4 \\ Kernel Patching and Cross-Compilation for RPi \\ Embedded Linux}
\author{Julia Desmazes \\ Michael Nissen}
\date{\today}
\maketitle
\bigskip

\section{Download / Set the good version of Linux sources}

\noindent }
To execute the tasks for this report, the first step is to get the Raspberry Pi cross-compilation tools, and specific version of the linux kernal that one wants to patch. This is going to be done using \code{git clone}, which ofcourse assumes a working installation of \texttt{git}.
\newline
To get the latest version of the kernel source code and the cross-compilation tools, the following commands with be copied to the terminal:
\begin{lstlisting}[language=shell]
	git clone https://github.com/raspberrypi/tools.git
	git cone -b rpi-4.4.y https://github.com/raspberrypi/linux.git
\end{lstlisting}
One now has to make sure to get the same configuration as is used by ones Raspberry Pi during the kernel compilation, which can be done by checking out the \texttt{git branch} that corresponds to the hash of the Linux kernel running on the Raspberry Pi.
\newline
This will be done by by running the shells script seen in Code Snippet \ref{lst:git_hash}. This shell script will grep the firmware hash found on the Raspberry pi, and then get the \texttt{git hash} from the github link relative to the hardware hash.
\begin{lstlisting}[language=shell, caption={Shell script to get Linux kernel git hash}, label={lst:git_hash}]
#!/bin/bash

PLATFORM=`uname -s` 

if [ ${PLATFORM} = "Darwin" ]
then
    CMD="gunzip -c"
elif [ ${PLATFORM} = "Linux" ]
then
    CMD="zcat"
else
    echo "Sorry, the platform ${PLATFORM} is not supported !!!"    
    exit -1
fi

FWHASH=`${CMD} /usr/share/doc/raspberrypi-bootloader/changelog.Debian.gz | grep -m 1 'as of' | awk '{print $NF}'`
#echo "Firmware Hashcode: fwhash = $FWHASH"

LINUXHASH=`wget -qO- https://raw.github.com/raspberrypi/firmware/$FWHASH/extra/git_hash`

echo "Linux Hashcode: linuxhash = $LINUXHASH"
\end{lstlisting}
When one has figured out the \texttt{git hash}, one can change directory to the linux git repository and checkout the respective brach by the following command: 
\begin{lstlisting}[language=shell]
cd linux
git checkout <git_hash>
\end{lstlisting}

\section*{The \code{grep}/\code{sed} and Ragular Expressions}

\noindent 
\textbf{1. Propose a \code{grep} command that displays user names in \texttt{/etc/passwd}
whom UID is multiple of 2?}
\newline
\newline
We use \code{grep} line by line on \texttt{/etc/passwd} and select all the UID's that finish by either 0,2,4,6,8. Then we pipe the output to cut where we delimite the output with \textbf{:} and select only the first match \code{cut -d: -f1}.
\newline
\newline
\begin{lstlisting}[language=bash]
grep '\^[\^:]*:[\^:]*:[0-9]*[02468]:.' /etc/passwd | cut -d: -f1
\end{lstlisting}

\noindent
\textbf{2. Propose a grep command that displays from the output of \code{ifconfig}, the WiFi IP address?}
\newline
\newline
To only display the wifi IP adresse we first only printed out the content of wlan0\footnote{Refers to the wifi hardware.}, then used \code{grep 'inet '} to select the line where the IPv4 IP addresse is displayed. Notice we added a space behind \texttt{inet}. This was to avoid matching inet6. Afterwards we piped the output to \code{awk \$2} in order to select only the second motif\footnote{By default seperation of \code{awk} is the space, this can be changed with the \code{-F} argument.}, here our IP.
\newline
\newline
\begin{lstlisting}[language=bash]
sudo ifconfig wlan0 | grep 'inet '| awk '{print \$2}'
\end{lstlisting}
\noindent
\section*{Shell Script}
\noindent
\textbf{Write a Shell script that displays the entries of \texttt{/etc/passwd}, using\code{sed} and \code{for} loops}
\newline
\newline
\texttt{Username: <username>, Password: encrypted, UID: <uid>, GID: <gid>, Home: <home\_directory>}
\newline
where \emph{i} is the line number of the corresponding <username> in \texttt{/etc/passwd}.
\newline
\newline
The script for this can be found in the \texttt{script} folder under the name \texttt{script3.sh}.
\newline
In this script we are using the \code{for} loop to read the file \texttt{/etc/passwd} line by line. Also we have a local value \code{i} to keep trake of the line number. For the \code{sed} commande we identifing substrings with our regex \code{\\(<match substring>\\)}and refrencing them latter with \code{\\<number>} pattern in the latter part of our commande where we re-define the output format.
%\footnote{We found out about this usage of sed with manual page \url{http://www.grymoire.com/Unix/Sed.htmluh-48}.} 
\begin{lstlisting}[language=bash,caption={RegEx in Shell Script}]
#!/bin/bash

#init i
i=1
#read line by line
for line in $(cat /etc/passwd)
do
    #increment i
    i=$((i+1))

    #parse and modify output stream with sed
    echo $i":"$line | sed  's/^\([^:]*\):\([^:]*\):\([^:]*\):\([0-9]*\):\([0-9]*\)::\([^:]*\).*/\1) Username: \2 ,Password: encrypted, UID: \4, GID: \5, Home: \6/'
done
\end{lstlisting}


\section*{GNU Make}

\noindent
In order to compile both nativly and cross-compile for out RaspberryPi we check at the start of our makefile for the variable \code{CROSS\_COMPILE}. If set, we compile with the cross compilation toolchain that we have stored in tools from our linux kernel compilation lab located in \texttt{/tools/arm-bcm2708/gcc-linaro-arm-linux-gnueabihf-raspbian-x64/bin/arm-linux-gnueabihf-}. If not, this variable will not be set and we will be using the machines standard compiler: gcc or g++ , depending on the value given to \code{TMPGCC} at the start of the makefile.
\newline
To access \code{CROSS\_COMPILE} in the makefile we have to make sure the variable is exported to the bash \code{export CROSS\_COMPILE}.
\newline
The rest of the makefile is very standard; we generate an object files based on ther .c files and link them together to create our executable main. To clean we select all object file \code{*.o} and delete them as well as our generated executable.

\begin{lstlisting}[language=bash,caption={MinMax Makefile}]
# if we wanted to cross compile to rpi we would use the toolchain we have downloaded for the kernel compilation : arm-linux-gnueabi-gcc, but to use it we would have to specify a path. In our case we have exported the path to the folder in the vaiable CROSS_COMPILE

TMPGCC=gcc

ifeq (${CROSS_COMPILE},'')
    echo "CROSS_COMPILE is not set";
    GCC=${TMPGCC}
else
    # to avoid recursice refrencing to GCC variable
    GCC=${CROSS_COMPILE}${TMPGCC}
endif

ADD=minmax
DEPS=minmax${HEADER_SUFFIX}

all: main

main: main.o minmax.o
	${GCC} -o main minmax.o main.o

main.o: main.c minmax.h
	${GCC} -c main.c

minmax.o: minmax.c minmax.h
	${GCC} -c minmax.c
clean:
	rm -f *.o main
\end{lstlisting}

\section*{Managing Process, System Call \code{fork()}}

\noindent
The system call \code{fork()} is used to create processes and returns a process ID. When calling \code{fork()}, it creates a new process which becomes a \textit{child} process of the caller. After a child process has been created, both the \textit{child} and \textit{parent} process will execute the next instruction following the \code{fork()} system call. Therefore, one has to distingush the difference between the two processes.
\newline
This can be done by testing the process ID returned by the system call.
\newline
\newline
If \code{fork()} returns a negative value, the creation of a child process was unsuccessful. It will return a zero to the newly created \textit{child} process, and it will return a possitive value, the \texttt{pid} of the \textit{child} process, to the parent.
\newline
\newline
Therefore it is simple to check what which is the child process - this can be seen in Code Snippet 1 below.
\begin{lstlisting}[language=c, caption=Managing system calls]
#include <stdio.h>
#include <sys/types.h>
#include <unistd.h>

int main() {

    // The process ID
    pid_t pid;

    // Store the process ID
    pid = fork();

    // Check for the process ID
    if (pid == 0) {

        // Child process

        // Create a new fork
        pid = fork();

        // Check for the process ID
        if (pid == 0) {

            // Grand child process
            printf("Child of Child");
        } else {

            // Child process
            printf("Child");
        }
    } else {

        // This is the parent process
        printf("Parent");
    }
}
\end{lstlisting}



\end{document}