\documentclass[a4paper,oneside,onecolumn]{article}
\usepackage[utf8x]{inputenc}
\usepackage[T1]{fontenc}
\usepackage{lmodern}
\usepackage{mathtools}
\usepackage{color}
\usepackage{siunitx}
\usepackage{microtype}
\usepackage{fancyhdr}
\usepackage{amsmath}
\usepackage{listings}
\usepackage{amssymb}
\usepackage{}
\usepackage[head=12pt,hmargin=2.5cm,vmargin=2.5cm]{geometry}

\renewcommand*\lstlistingname{Code Snippet}

\definecolor{codegreen}{rgb}{0,0.6,0}
\definecolor{codegray}{rgb}{0.5,0.5,0.5}
\definecolor{codepurple}{rgb}{0.58,0,0.82}
\definecolor{backcolour}{rgb}{0.95,0.95,0.92}

\lstdefinestyle{mystyle}{
    backgroundcolor=\color{backcolour},   
    commentstyle=\color{codegreen},
    keywordstyle=\color{magenta},
    numberstyle=\tiny\color{codegray},
    stringstyle=\color{codepurple},
    basicstyle=\footnotesize,
    breakatwhitespace=false,         
    breaklines=true,                 
    captionpos=b,                    
    keepspaces=true,                 
    numbers=left,                    
    numbersep=5pt,                  
    showspaces=false,                
    showstringspaces=false,
    showtabs=false,                  
    tabsize=2
}
 
\lstset{style=mystyle}

\DisableLigatures[>,<]{encoding = T1,family=tt*} %

\definecolor{codegray}{gray}{0.9}
\newcommand{\code}[1]{\colorbox{codegray}{\texttt{#1}}}

\pagestyle{fancy}

\begin{document}

\title{Lab 5 \\ GPIO and Interrupt Management Using Kernel Modules \\ Embedded Linux}
\author{Michael Nissen}
\date{\today}
\maketitle
\bigskip

\section{Configurable LED Blink Module}

\subsection{Simple GPIO Blink Module}

In this section, a simple module was created to display a start message and blink a LED using a timer. This timer will start with an initial period of 1 second when it is loaded. When the module is unloaded, it will display a termination message and stop blinking the LED. The code in Code Snippet \ref{lst:init_blink_simple} contains the code to handle the initialisation of the LED blinking.
\newline
\newline
Firstly, the initialisation function, \code{\_\_init}, will print a message saying that it is loading the module, and that it will blink with a period of 1 second. It will then start by setting a direction for the output of a specified \texttt{GPIO} pin and the initial configuration as the LED being set low. This is done with the \code{gpio\_direction\_output()} function. If an error occurs during this operation, this will be printed to the terminal. If everything is successfull, a timer will be initialised and the function to be called on expiration is defined as \code{my\_timer.function}. The expiration time of the timer is defined with \code{my\_timer.expires}, which is set to 1 second. Finally the timer will be started, and this will now run until the module is unloaded.

\begin{lstlisting}[language=c, label={lst:init_blink_simple}, caption={Initialisation Function}]
#define LED_ON 0x01
#define LED_OFF 0x00

//gpio defined
#define GPIO_PIN_NUMB 4

static int __init start_blink(void) {

    int err;
    
    printk(KERN_INFO "Loading blink led module, blinking shall start at a rate of 1 second.\n");
    
    //init gpio
    err = gpio_direction_output(GPIO_PIN_NUMB, LED_OFF);

    if(err < 0) {

        printk(KERN_ALERT "Could not configure pin to output init failed");
        return -1;

        printk(KERN_ALERT"Could not configure pin to output init failed");
        return -1;
    }

    //init timer
    init_timer(&my_timer);
    my_timer.function = my_timer_func;
    my_timer.data = (unsigned long)&kbledstatus;
    my_timer.expires = jiffies + HZ;
    add_timer(&my_timer);
    return 0;
}
\end{lstlisting}

When the module is unloaded, the \code{\_\_exit} macro function is run. This will start off by printing a status message saying that the module is being unloaded and that the blinking will siese to happen. It will then delete the timer and turn off the \texttt{GPIO} pin. This can be seen in Code Snippet \ref{lst:exit_blink_simple} below. 

\begin{lstlisting}[language=c, label={lst:exit_blink_simple}, caption={Initialisation Function}]
static void __exit stop_blink(void) {

    printk(KERN_INFO "Blink module is beeing unloaded , blinking will stop\n");
    
    //delete timer
    del_timer(&my_timer);

    //turn off led
    gpio_set_value(GPIO_PIN_NUMB, LED_OFF);
}
\end{lstlisting}

When the timer expires, the code seen in Code Snippet \ref{lst:blink_simple} will be executed. This will print that it is blinking the LED and then check the status of the LED. If the LED is set \texttt{HIGH} it will set the LED \texttt{LOW}, otherwise it will set it \texttt{HIGH}. This is done by the shorthand \\ \code{status == LED\_ON ? <if statement is true> : <statement is false>}. It will then set the appropriate value to the \texttt{GPIO} pin, and set a new expiration time for the timer and start it over again.

\begin{lstlisting}[language=c, label={lst:blink_simple}, caption={Initialisation Function}]
static void my_timer_func(unsigned long ptr) {

    printk(KERN_INFO "Blink led");
    int *pstatus = (int *)ptr;
    *pstatus= ((*pstatus == LED_ON) ? LED_OFF : LED_ON);
    gpio_set_value(GPIO_PIN_NUMB, *pstatus);
    my_timer.expires = jiffies + HZ;
    add_timer(&my_timer);
}
\end{lstlisting}

\subsection{sysfs-based User Interface}

\noindent
The goal here is to create a \texttt{sysfs}-based user interface that allows the user to change the LED blinking frequency at runtime. When the module is loading, it is initially going to display a message and blink the LED using a default timer with a period of 1 second.
\newline
In the Code Snippet \ref{lst:initial_blink} below, we can see the \code{\_\_init} function, that is a macro used to mark some functions or initialized data as an `initialisation' function. The kernel can then take this as a hint that the function is used only during the initialisation phase, and free up used memory resources after. Essentially, the kernel build system looks for all of the functions flagged with \code{\_\_init}, across all of the pieces of the kernel, and arranges them so that they will all be in the same block of memory. Then, when the kernel boots, it can free that one block of memory at once.
\newline
\newline
The code starts by printing a message to the terminal. It is then requsting a single \texttt{GPIO} pin with an initial setup using the \code{gpio\_request\_one}. As seen in the comments for the code, it takes the arguments of the \texttt{GPIO} pin number, a flag the specified the initial \texttt{GPIO} configuration, and a label with a literal description string of this \texttt{GPIO}. The rusult is stored in a global \code{result} variable, that we can use to check if the request and initialisation of the \texttt{GPIO} pin was successful or not. If \code{result} is not 0, it will print error message to the terminal saying that it could not initialise the \texttt{GPIO} pin, otherwise it will print a statement saying it was successfull.
\newline
We then initialise our timer by \code{init\_timer(\&my\_timer)}, set an expiration time (1 second), and define what function will be executed once the timer expires. We want to execute our \code{blink\_timer} function, that handles the blinking. We that print out a status message saying that the timer has been initialised, and start the timer.
\newline
The timer LED should now be starting with an initial configuration of \texttt{HIGH}, and blinking on and off every 1 second.
\begin{lstlisting}[language=c, label={lst:initial_blink}, caption={Initial Configuration of LED}]
static int __init start_blink(void) {

    printk("Blink :: start module :\n");

    // Request a single GPIO with initial configuration HIGH: <PIN, FLAGS, label>
    result = gpio_request_one(GPIO_PIN_NUMB, GPIOF_OUT_INIT_HIGH, "sysfs");

    if (result) {

        printk(KERN_INFO "ERROR :: Failed to initialise the GPIO PIN: %d\n", GPIO_PIN_NUMB);

        return result;
    } else {

        printk(KERN_INFO "SUCCESS :: GPIO PIN was successfully allocated\n");
    }

    // Initialise the timer
    init_timer(&my_timer);

    // Set expiration time in jiffies - convert period to milliseconds
    my_timer->expires = jiffies + HZ*period/1000;

    // Define the function to be executed upon timer expire
    my_timer->function = blink_timer;

    my_timer->data = 0;

    printk(KERN_INFO "STATUS :: The timer has been initialised\n");

    // Start the timer
    add_timer(&my_timer);

    ...
}
\end{lstlisting}

Now that we have our initial configuration up and running, we want to register a device class. A class is a higher-level view of a device that abstracts out low-level implementation details and can be found in \texttt{syfs} under the \texttt{/sys/class/} folder. In many cases, the class subsystem is the best way of exporting information to the user space. When a subsystem creates a class, it owns the class entirely, so there is no need to worry about which module owns the attributes found there.
\newline
A class is defined by an instance of \code{struct} class. Each class needs a unique name, which is how the class appears under \texttt{/sys/class}.
\newline
\newline
If we take a look at Code Snippet \ref{lst:device}, we start off by creating a \code{struct} class structure using the \code{class\_create} function. We pass in the module that owns the class, and a unique string for the name of the class - this is defined in the \code{\#define CLASS\_NAME} at the start of the code. We then check if there was an error creating the class - if there was, we print out an error message saying that the creation of the device class has failed and return the error, otherwise we print a success statement, saying it was created.
\newline
\newline
We then go on to register the device object with \texttt{sysfs} using the \code{device\_create} which takes as arguments, the class it should be registered to, the parent struct device, the \texttt{dev\_t} to be added, and a unique string for the device name. As name of the class has been defined as \texttt{external\_led} and the name of the device \texttt{blink}, the device object can be found under \texttt{/sys/class/external\_led/blink}.
\newline
As usual, we check if the device object was created successfully and if not we destroy the \code{struct} class structure with the \code{class\_destroy} and print the error message, else we print a success message.

\begin{lstlisting}[language=c, label={lst:device}, caption={Registering Device Class and Object}]
#define DEVICE_NAME "blink"
#define CLASS_NAME  "external_led"

static int __init start_blink(void) {

    ...

    // Create a struct class structure <Module that owns the class, String for the name of the class>
    external_leds = class_create(THIS_MODULE, CLASS_NAME);

    if (IS_ERR(external_leds)) {

        printk(KERN_ERR "ERROR :: Failed to create device class\n");

        return PTR_ERR(external_leds);
    } else {

        printk(KERN_INFO "SUCCESS :: EXTERNAL_LEDS :: Device class was successfully created\n");
    }

    // Create a device and register it with sysfs:
    // <Class it should be registered to, Parent struct device, dev_t to be added, String for device name>
    blink = device_create(external_leds, NULL, 0, NULL, DEVICE_NAME);

    if (IS_ERR(blink)) {

        // Destroy the struct class structure if the was an error
        class_destroy(external_leds);

        printk(KERN_ERR "ERROR :: Failed to create the device\n");

        return PTR_ERR(blink);
    } else {

        printk(KERN_INFO "SUCCESS :: Device class created successfully\n");
    }

    ...
}
\end{lstlisting}

We now go on to create a \texttt{sysfs} attribute file for the device using the \code{device\_create\_file()} which takes the device created with \code{device\_create()} and the device attribute descriptor as arguments. The device attribute descritor is a function, The device attribute descriptor is a file, that if an interger is writton to it, it can later be read out. A simple attribute has no means by which it can be read or written to - it needs wrapper routines for reading and writing. For this purpose, \texttt{kobjects} defines a special structure \code{struct kobj\_attribute} where the first argument represents the name of the attribute file, the second argument is the file permission (0644 which is read/write for the owning user and read for group/other), the third argument is the function to be called when the file is read, and the last argument is the function to be called when the file is written.
\newline
\newline
We define the name, permission, and functions to be executed when using the attribute file, and then we create the device file passing in the device and the pointer to the address of the the \code{kobj\_attribute}. We then check if the device file was succesfully created or not, and print out a relevant message depending on the result.
\newline
As we gave the file the name \texttt{period}, it can be written to by echoing an integer value into the file placed in \texttt{/sys/class/external\_led/blink/period}.
\newline
This can be seen in Code Snippet \ref{lst:sysfs}, below.

\begin{lstlisting}[language=c, label={lst:sysfs}, caption={Creating a \texttt{sysfs} Attribute File}]
// Device attribute descriptor: <attribute name, permission, function, function>
static struct kobj_attribute period_attr = __ATTR(period, 0644, NULL, store_period);

static int __init start_blink(void) {

    ...

    // Create sysfs attribute file for device: <device, device attribute descriptor>
    result = device_create_file(blink, &period_attr);

    if (result) {

        printk(KERN_ERR "ERROR :: Failed to create the sysfs file\n");
    } else {

        printk(KERN_INFO "SUCCESS :: sysfs file successfully created\n");
    }

    ...

}
\end{lstlisting}

By unloading the module, the extention shall delete the \texttt{sysfs}-based interface, display a termination message, delete the timer, stop blinking the LED, and release the \texttt{GPIO} request. All of this is done using the \code{\_\_exit} macro. The \code{\_\_exit} macro is used to deallocate any and all resources it has acquired, as well as any hardware it may have activated. The exit routine is the last chance that the module has to perform these operations, or it may leave the system in an unstable state.
\newline
\newline
Taking a look at the code in Code Snippet \ref{lst:exit}, we start by setting a value of low on the \texttt{GPIO} pin to turn off the LED, we free up the \texttt{GPIO} pin, deactivate the timer, remove and unregister the device class and object, destroying the class structure and removing the \texttt{sysfs} attribute file from its own method.

\begin{lstlisting}[language=c, label={lst:exit}, caption={Unloading of the Module}]
static void __exit stop_blink(void) {

    // Turn off LED
    gpio_set_value(GPIO_PIN_NUMB, LED_OFF);

    printk(KERN_INFO "BLINK :: LED turned OFF\n");

    // Free the GPIO PIN
    gpio_free(GPIO_PIN_NUMB);

    printk(KERN_INFO "BLINK :: GPIO freed\n");

    // Deactivate the timer
    del_timer(&my_timer);

    printk(KERN_INFO "BLINK :: Timer deactivated\n");

    // Remove the device created by device_create
    device_destroy(external_leds, 0);

    // Unregister the previously registered class
    class_unregister(external_leds);

    // Destroy the struct class structure
    class_destroy(external_leds);

    // Remove sysfs attribute file from its own method
    device_remove_file(blink, &period_attr);

    printk(KERN_INFO "BLINK :: The module has been unloaded\n");
}
\end{lstlisting}

Last but not least, we have the function that is to be executed when the timer expires. This function will perform a simple investigation of the initial value of the \texttt{jiffies}. \texttt{jiffies} is a global variable which only usage is to store the number of ticks occured since system start-up. On kernel boot-up, jiffies is initialised to a special initial value, and it is incremented by one for each timer interrupt. Since there are \texttt{HZ} ticks occuring in one second, there are \texttt{HZ jiffies} in a second.
\newline
The function will then investigate the status of the LED using the global variable \code{led\_status} - is it on or is it off, and determine the correct behavior based on the status. If the LED is on, we set the value of the \texttt{GPIO} pin to 0 with \code{gpio\_set\_value()}, set the \code{led\_status} to 0 and print a status message saying that the LED was turned off. If the LED is off, the opposite is happening.
\newline
The expiration time of the timer is then set to the value specified in the period \texttt{sysfs} attribute file, and the the timer is started again. This can be seen in Code Snippet \ref{lst:blink_timer}, below.

\begin{lstlisting}[language=c, label={lst:blink_timer}, caption={The Blink Function to be Executed on Timer End}]
unsigned int led_status = LED_ON;

static void blink_timer() {

    // Investigate initial value of jiffies
    printk(KERN_INFO "Jiffies %lu :", jiffies);

    // If the LED is on - turn it off / else turn it on
    if (led_status) {

        gpio_set_value(GPIO_PIN_NUMB, LED_OFF);
        led_status = 0;
        printk(KERN_INFO "SET :: LED OFF\n");
    } else {

        gpio_set_value(GPIO_PIN_NUMB, LED_ON);
        led_status = 1;
        printk(KERN_INFO "SET :: LED ON\n");
    }

    // Set expiration time in jiffies - convert period to milliseconds
    my_timer->expires = jiffies + HZ * period / 1000;

    // Start the timer
    add_timer(&my_timer);
}
\end{lstlisting}

\section{Button-Controlled LED Blink Module}



\begin{thebibliography}{9}
\bibitem{linuxgit} 
Torvalds: Linux Kernel Source Tree,
\\\texttt{https://github.com/torvalds/linux}

\bibitem{linuxdevicedrivers} 
J. Corbet, A. Rubini, G. Kroah-Hartman: Linux Device Drivers,
\\\texttt{https://lwn.net/Kernel/LDD3/}

\bibitem{linuxkernel} 
The Kernel Development Community: The Linux Kernel,
\\\texttt{https://01.org/linuxgraphics/gfx-docs/drm/index.html}

\bibitem{linuxinsides} 
0xAX: Linux Inside,
\\\texttt{https://www.gitbook.com/book/0xax/linux-insides/details}
\end{thebibliography}


\end{document}