\documentclass[10pt,a4paper]{article}
\usepackage[utf8]{inputenc}
\usepackage[francais]{babel}
\usepackage[T1]{fontenc}
\usepackage{graphicx}
\author{Julia Desmazes and Michael Nissen}
\title{Lab1}
\begin{document}
\paragraph{1. How the IP address of your RPi can be heard before authentication ? In which configuration file
in /etc this setting is made ?}
\vspace{0.5cm}
We call the script "/etc/rc.local" on boot .
\paragraph{2. Which software is used for the vocal synthesis of the IP ?}\vspace{0.5cm}

The vocal synthesis uses "festival" with option --tts\footnote{To systemsis text files to speatch.} and pipes the value of our ip to the vocal systesis.
\paragraph{3. Which configuration should be made to modify this setting by a new one allowing the sending of
the IP to your personal mailbox ?}\vspace{0.5cm}

Before starting we would need to need to install and configure a mailing service we would chose "mailutils" in order to use "mail". We would simply have to add a line in the script a usage of "mail":
\textbf{mail -s <subject of mail> <my mail address> < < < \$\{ <variable containing our IP>\} (or) hostname -I }
\paragraph{4. Using the command ifconfig, retrieve the IP address of your RPi.}\vspace{0.5cm}

ifconfig wlan0 gets 
\paragraph{5. How to scan the different WiFi networks detected by your dongle?}\vspace{0.5cm}

iwlist wlan0 scan | grep ESSID
\paragraph{6. Describe the content of the network configuration files /etc/wpa.conf and /etc/network/interfaces
and guess the relationship between them.}\vspace{0.5cm}

wpa.conf lists saved network configurations with the name, ssid and hash of the password, wheras interfaces lists all the the possible connection to the diffrent interfaces with there valide parameters. We can act upon the default behaviours regarding connecting to networks.
\paragraph{7. How the setting of the above files can be changed in order to make your RPi recognizes your home
WiFi network? If you do it, don’t forget to make backups !}\vspace{0.5cm}

You would have to add an new configuration under network with ssid , password next to \#psk and our security key hash if we hade one under psk. This last would depend on the security our local networks security configuration, you would have to use the propur hash be it md5, shasum5 , ect ...
\paragraph{8. By which command the pre-configured image was created after the modification of the official lite
version of Raspbian provided by the Raspberry Pi foundation?}\vspace{0.5cm}

We would use the "dd" commande to create a bootable image with usage :
\textbf{dd if=<input dir> of=<file to create with .img format> count=<size of file in Mbs>} .\vspace{0.5cm}

Later you can additionaly use \textbf{mkfs <filesystem> -F <file to format>} to format your created file for a diffrent file system.
\end{document}
