\documentclass[a4paper,oneside,onecolumn]{article}
\usepackage[utf8x]{inputenc}
\usepackage{mathtools}
\usepackage{color}
\usepackage{siunitx}
\usepackage{microtype}
\usepackage{fancyhdr}
\usepackage{amsmath}
\usepackage{listings}
\usepackage[T1]{fontenc}
\usepackage{amssymb}
\usepackage{}
\usepackage[head=12pt,hmargin=2.5cm,vmargin=2.5cm]{geometry}

\renewcommand*\lstlistingname{Code Snippet}

\definecolor{codegreen}{rgb}{0,0.6,0}
\definecolor{codegray}{rgb}{0.5,0.5,0.5}
\definecolor{codepurple}{rgb}{0.58,0,0.82}
\definecolor{backcolour}{rgb}{0.95,0.95,0.92}

\lstdefinestyle{mystyle}{
    backgroundcolor=\color{backcolour},   
    commentstyle=\color{codegreen},
    keywordstyle=\color{magenta},
    numberstyle=\tiny\color{codegray},
    stringstyle=\color{codepurple},
    basicstyle=\footnotesize,
    breakatwhitespace=false,         
    breaklines=true,                 
    captionpos=b,                    
    keepspaces=true,                 
    numbers=left,                    
    numbersep=5pt,                  
    showspaces=false,                
    showstringspaces=false,
    showtabs=false,                  
    tabsize=2
}
 
\lstset{style=mystyle}

\DisableLigatures[>,<]{encoding = T1,family=tt*} %

\definecolor{codegray}{gray}{0.9}
\newcommand{\code}[1]{\colorbox{codegray}{\texttt{#1}}}

\pagestyle{fancy}

\begin{document}

\title{Test Lab \\ Linux Shell / GNU Make \\ Embedded Linux}
\author{Julia Desmazes \\ Michael Nissen}
\date{\today}
\maketitle
\bigskip

\section*{Command line, environment variables, and user permissions}

\noindent 
\textbf{1. In the bash prompt, what is the meaning of the character \~{} ?}
\newline
\newline
The tilde (\~{}) character refers to the users home directory. The full path is \texttt{/home/user}.
\newline
\newline
\noindent
\textbf{2. Explain the behaviour of running in the order \code{VV=3}, \code{export VV}, \code{bash}, \code{unset VV}, \code{exit} and finally \code{echo \$VV}?}
\newline
\newline
Environment variables is a value, named at runtime, that can affect the way running processes behave. A running process will be able to query the value of an environment variable for example, to discover a location to store files. When creating an environment variable, it will persist in the shell in which it was initalised.
\newline
When exporting the creatied environment variable by using \code{export VARNAME}, it sets the environment variable not only for current shell, but all other processes started from that shell.
\begin{itemize}
	\item Set local variable VV to 3
	\item Export local variable as environment variable
	\item Enter the bash shell
	\item Unest the local variable
	\item Exit the bash shell
	\item Display the environment variable
\end{itemize}
\noindent
\textbf{3. How to run \texttt{/home/user/ls} instead of \texttt{/usr/bin/ls} automatically by typing \code{ls} without changin the behaviour of the other commands?}
\newline
\newline
An easy way to achive this kind of behaviour, is to create a shell alias. A shell alias is simply a shortcut to reference a command. It can be used to avoid typing long commands or as a mean to run increase efficiency.
In this example, we want to run the \texttt{/home/user/ls} command instead of the buildin one located in \texttt{/usr/bin/ls}. As the \texttt{/usr/bin} folder is added to the path of the shell, one is able to run any script from that folder by simply entering the name of that script.
\newline
As we don't necessarily want to add the \texttt{home/user} folder to the path, we can create an alias to the specific file we need. 
\newline
\newline
One can create an alias using echo and redirecting it to the shell configuration file: 
\newline
\code{echo alias 'ls="/home/user/ls"' >> .zshrc} where the last file is your shell configuration file.
\newline
\newline
\noindent
\textbf{4. How to provide \texttt{file.txt} as input of the command \code{flex} and copy its \texttt{stderr} to \texttt{resu.txt}?}
\newline
\newline
3 different streams. Use 2> to redirect the content on stderr. (-p writes the performance report to stdderr)
\newline
\code{flex -p file.txt 2> resu.txt }
\newline
\newline
\noindent
\textbf{5. Propose a command that displays the middle line of \texttt{/etc/passwd} ?}
\newline
\newline
In this commande we are usign \code{wc -l} to get the line count of \texttt{/etc/passwd} and storing it in a bash variable named NUM. Then we are dividing NUM by two \code{expr \$NUM /2} and start grabbing the line starting from this new line count with \code{head -n \$NUM /etc/passwd}. In order to only grabe one line from this line on we pipe the output to \code{tail -1}, witch grabe only the first line.
\newline
\begin{lstlisting}[language=bash,caption={bash version}]NUM=\$(wc -l </etc/passwd
x)
\& \& NUM=
\$ (expr 
\$ NUM / 2 )
\&\& head -n 
 \$ NUM /etc/passwd
 | tail -1
\end{lstlisting}
\noindent
\textbf{6. How to change the owner of a first file to the owner of a second one by using a command substitution based on the command \code{ls} and \code{cut} ?}
\newline
\newline
Easy way is to just use the build in functionality of \code{chown}. No need to reinvent the wheel:
\newline
\code{chown --reference=result.txt file.txt}

\section*{The \code{grep}/\code{sed} and Ragular Expressions}

\noindent 
\textbf{1. Propose a \code{grep} command that displays user names in \texttt{/etc/passwd}
whom UID is multiple of 2?}
\newline
\newline
We use \code{grep} line by line on \texttt{/etc/passwd} and select all the UID's that finish by either 0,2,4,6,8. Then we pipe the output to cut where we delimite the output with \textbf{:} and select only the first match \code{cut -d: -f1}.
\begin{lstlisting}[language=bash,caption={bash version}]
grep '\^[\^:]*:[\^:]*:[0-9]*[02468]:.' /etc/passwd | cut -d: -f1)
\end{lstlisting}
\noindent
\textbf{2. Propose a grep command that displays  from the output of \code{ifconfig}
, the WiFi IP address?}
\newline
\newline
To only display the wifi IP adresse we first only printed out the content of wlan0\footnote{Refers to the wifi hardware.}, then used \code{grep 'inet } to select the line where the ipv4 IP addresse is displayed. Notice we added a space behinde inet, this was to avoid matching inet6. Afterwards we piped the output to \code{awk \$2} in order to select only the second motif\footnote{By default seperation of \code{awk} is the space, this can be changed with the -F argument.}, here our IP.
\begin{lstlisting}[language=bash,caption={bash version}]
sudo ifconfig wlan0 | grep 'inet '| awk '{print \$2}')
\end{lstlisting}

\noindent
\section*{Shell Script}
\noindent
\textbf{Write a Shell script that displays the entries of
\texttt{/etc/passwd}
, using
\code{sed}
and
\code{for}
loops, as follows:
\newline
i) Username:  <username>, Passwrd:  encrypted, UID: <uid>, GID: <gid>, Home:  <home
directory>
\newline
where i
is the line number of the corresponding
<username>
in
\texttt{
/etc/passwd}
}
\newline
\newline
The script for this can be found in the \texttt{script} folder under the name \texttt{script3.sh}.
\newline
In this script we are using the \code{for} loop to read the file \texttt{/etc/passwd} line by line. Also we have a local value 
\code{i} to keep trake of the line number.  For out \code{sed} commande we identifing substrings with our regex and refrencing them latter with \code{\\ <number>} pattern
%\footnote{We found out about this usage of sed with manual page \url{http://www.grymoire.com/Unix/Sed.htmluh-48}.} 
in the latter part of our commande where we re-define the output format.
\begin{lstlisting}[language=bash,caption={bash version}]
#!/bin/bash

#init i
i=1
#read line by line
for line in $(cat /etc/passwd)
do
    i=$((i+1))#increment i
    #parse and modify output stream with sed
    echo $i":"$line | sed  's/^\([^:]*\):\([^:]*\):\([^:]*\):\([0-9]*\):\([0-9]*\)::\([^:]*\).*/\1) Username: \2 ,Password: encrypted, UID: \4, GID: \5, Home: \6/'
done
\end{lstlisting}

\noindent
\section*{GNU Make}
In order to compile both nativly and cross-compile for out rassberypi we check at the beging of our makefile for the variable \code{CROSS COMPILE}. If it is set we compile with the cross compilation toolchain that we have stored in tools from our linux kernel compilation lab located in \texttt{*/kernel labs/tools/arm-bcm2708/gcc-linaro-arm-linux-gnueabihf-raspbian-x64/bin/arm-linux-gnueabihf-}. If not this variable will not be set and we will be using the machines standard compiler: gcc or g++ , depending on the value given to \code{TMPGCC} at the begging of the makefile.\newline
To access \code{CROSS COMPILE} in the makefile we have to make sure the variable is exported to the bash \code{export CROSS COMPILE}.
\newline
The rest of the makefile is very standard : we generate object files based on there .c files and link them together to create our executable main. To clean we select all object file \code{*.o} and delete them as well as our generated executable.

\begin{lstlisting}[language=bash,caption={bash version}]
TMPGCC=gcc
#if we wanted to cross compile to rpi we would use the toolchain we have downloaded for the kernel compilation : arm-linux-gnueabi-gcc, but to use it we would have to specify a path. In our case we have exported the path to the folder in the vaiable CROSS_COMPILE
ifeq (${CROSS_COMPILE},'')
echo "CROSS_COMPILE is not set";
GCC=${TMPGCC}
else
#to avoid recursice refrencing to GCC variable
GCC=${CROSS_COMPILE}${TMPGCC}
endif
ADD=minmax
DEPS=minmax${HEADER_SUFFIX}
all: main

main: main.o minmax.o
	${GCC} -o main minmax.o main.o

main.o: main.c minmax.h
	${GCC} -c main.c

minmax.o: minmax.c minmax.h
	${GCC} -c minmax.c
clean:
	rm -f *.o main

\end{lstlisting}


\noindent
\section*{Managing Process, System Call \code{fork()}}

\noindent
The system call \code{fork()} is used to create processes and returns a process ID. When calling \code{fork()}, it creates a new process which becomes a \textit{child} process of the caller. After a child process has been created, both the \textit{child} and \textit{parent} process will execute the next instruction following the \code{fork()} system call. Therefore, one has to distingush the difference between the two processes.
\newline
This can be done by testing the process ID returned by the system call.
\newline
\newline
If \code{fork()} returns a negative value, the creation of a child process was unsuccessful. It will return a zero to the newly created \textit{child} process, and it will return a possitive value, the \texttt{pid} of the \textit{child} process, to the parent.
\newline
\newline
Therefore it is simple to check what which is the child process - this can be seen in Code Snippet 1 below.
\begin{lstlisting}[language=c, caption=Managing system calls]
#include <stdio.h>
#include <sys/types.h>
#include <unistd.h>

int main() {

    // The process ID
    pid_t pid;

    // Store the process ID
    pid = fork();

    // Check for the process ID
    if (pid == 0) {

        // Child process

        // Create a new fork
        pid = fork();

        // Check for the process ID
        if (pid == 0) {

            // Grand child process
            printf("Child of Child");
        } else {

            // Child process
            printf("Child");
        }
    } else {

        // This is the parent process
        printf("Parent");
    }
}
\end{lstlisting}



\end{document}